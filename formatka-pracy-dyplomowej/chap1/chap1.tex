% ********** Rozdzia³ 1 **********

\chapter{Wstęp}
\label{sec:chapter1}


Praca dyplomowa ma na celu sprawdzenie umiejętności wykonywania samodzielnej pracy przez studenta oraz umiejętności rozwiązywania stawianych przed nim problemów.

Niniejszy dokument, którego struktura oraz sposób formatowania odpowiada pracy dyplomowej, został utworzony na podstawie zaleceń, dotyczących pisania prac dyplomowych.

Praca inżynierska powinna stanowić opis rozwiązania zadania inżynierskiego, zawierający element projektowy lub eksperymentalny, z opisem uzyskanych wyników. Praca magisterska powinna zawierać: element projektowy lub eksperymentalny, szczegółową analizę i krytyczne odniesienie do rozwiązywanego problemu (z uwzględnieniem literatury), element wykorzystujący modele matematyczne z oceną ich przydatności do rozwiązania podjętego problemu oraz uzasadnioną ocenę oryginalności i skuteczności rozwiązania podjętego problemu.
We wstępie pracy dyplomowej należy zamieścić krótkie wprowadzenie do tematyki omawianej w pracy.

\section{Cel pracy}
\label{sec:1:chapter1}
Powyższy podrozdział powinien zawierać omówienie celów pracy oraz zakresu problemów i zadań, jakie mają zostać rozwiązane w ramach realizowanej przez studenta pracy dyplomowej.



% ********** Koniec rozdzia³u **********
