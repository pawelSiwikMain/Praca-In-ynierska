% ********** Rozdział ł **********
\chapter{Prezentacja i omówienie wyników pracy}
\label{sec:chapterł}


Praca stanowi formalny dokument naukowy, dlatego też musi być napisana stylem odpowiednim dla pracy naukowej. W pracy należy bezwzględnie unikać:
\begin{itemize}
    \item Formy osobowej czasowników  - należy użyć formy np.:  "\textit{Wykonano szereg pomiarów...}", "\textit{Zdaniem autora oznacza to, ...}"),
    \item słów oraz sformułowań gwarowych i żargonowych, itp.,
    \item sformułowań niegramatycznych.
\end{itemize}


Niniejszy rozdział powinien zawierać prezentację i omówienie efektów pracy, 
w przypadku programów — opis zastosowanych algorytmów, wykorzystane techniki, rozwiązania, sposób instalowania i posługiwania się programem, rysunki przedstawiające wygląd poszczególnych ekranów, etc.


Należy pamiętać, że właśnie ten rozdział jest najważniejszą częścią pracy, ponieważ  powinien  zawierać  (opisywać)  faktycznie  zrealizowane  przez  Autora prace. Powinno to też znaleźć wyraz w objętości — najbardziej obszerny rozdział pracy.






% ********** Koniec rozdziału **********
